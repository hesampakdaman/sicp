% Created 2020-07-09 Thu 18:59
% Intended LaTeX compiler: pdflatex
\documentclass[11pt]{article}
\usepackage[utf8]{inputenc}
\usepackage[T1]{fontenc}
\usepackage{graphicx}
\usepackage{grffile}
\usepackage{longtable}
\usepackage{wrapfig}
\usepackage{rotating}
\usepackage[normalem]{ulem}
\usepackage{amsmath}
\usepackage{textcomp}
\usepackage{amssymb}
\usepackage{capt-of}
\usepackage{hyperref}
\date{\today}
\title{}
\hypersetup{
 pdfauthor={},
 pdftitle={},
 pdfkeywords={},
 pdfsubject={},
 pdfcreator={Emacs 26.3 (Org mode 9.1.9)}, 
 pdflang={English}}
\begin{document}

\noindent\textbf{Exercise 1.35}: We show that the golden ratio $\varphi$ is a fixed point to $x \mapsto 1 + 1/x$.
If $\varphi$ is a fixed point, then successive function applications should not alter the value.
Let $f(x) = 1 + 1/x$, then
\begin{align*}
  f(\varphi) = 1 + \frac{1}{\varphi} = \frac{\varphi + 1}{\varphi} = \frac{\varphi^2}{\varphi} = \varphi,
\end{align*}
where we have used the fact $\varphi^2 = \varphi + 1$.
Clearly, applying $f$ again will result in the same value.

\hfill$\square$

\end{document}