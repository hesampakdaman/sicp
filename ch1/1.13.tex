% Created 2020-07-09 Thu 18:59
% Intended LaTeX compiler: pdflatex
\documentclass[11pt]{article}
\usepackage[utf8]{inputenc}
\usepackage[T1]{fontenc}
\usepackage{graphicx}
\usepackage{grffile}
\usepackage{longtable}
\usepackage{wrapfig}
\usepackage{rotating}
\usepackage[normalem]{ulem}
\usepackage{amsmath}
\usepackage{textcomp}
\usepackage{amssymb}
\usepackage{capt-of}
\usepackage{hyperref}
\date{\today}
\title{}
\hypersetup{
 pdfauthor={},
 pdftitle={},
 pdfkeywords={},
 pdfsubject={},
 pdfcreator={Emacs 26.3 (Org mode 9.1.9)}, 
 pdflang={English}}
\begin{document}

\noindent\textbf{Exercise 1.13}: First we show that \(\mathrm{Fib}(n) = ({\varphi^n - \psi^n})/\sqrt{5}\) using induction. It is clear that
\(\mathrm{Fib}(1) = (\varphi-\psi)/\sqrt{5} = 1\). Now we show that this holds for \(k+1\)
\begin{align*}
        \mathrm{Fib}(k+1) &= \mathrm{Fib}(k) + \mathrm{Fib}(k-1)
        = \frac{\varphi^{k} - \psi^{k}}{\sqrt{5}} + \frac{\varphi^{k-1} - \psi^{k-1}}{\sqrt{5}}     \\
        &= \frac{ (\varphi+1)\varphi^{k-1} - (\psi+1)\psi^{k-1} }{ \sqrt{5} }
        = \frac{ \varphi^2\varphi^{k-1} - \psi^2\psi^{k-1} }{ \sqrt{5} } \\
        &= \frac{ \varphi^{k+1} - \psi^{k+1} }{ \sqrt{5} },
\end{align*}
where used the fact that both \(\varphi\) and \(\psi\) satisfy \(x^2=x+1\). Using the result above,
we get that
\begin{align*}
        \Bigl| \frac{ \varphi^n } {\sqrt{5}} - \mathrm{ Fib }(n) \Bigr| &=
        \Bigl| \frac{ \psi^n } {\sqrt{5}} \Bigr| =
        \Bigl| \frac{ (1-\sqrt{5})^n } {2^n\sqrt{5}} \Bigr| <
        \frac{1}{2} \cdot \Bigl| {\frac{ 1-\sqrt{5} } {2}} \Bigr|^n
        < \frac{1}{2} \cdot \Bigl| {\frac{ 1-3 } {2}} \Bigr|^n \\
        & \leq \frac{1}{2}.
\end{align*}
Suppose there exists another integer \(m\) that is closer to \(\varphi^n / \sqrt{5}\). Then
\begin{align*}
        \Bigl| \mathrm{Fib}(n) - m \Bigr| &= \Bigl| \mathrm{Fib}(n) - \frac{ \varphi^n } {\sqrt{5}}
        + \frac{ \varphi^n } {\sqrt{5}} - m \Bigr| \\
        & \leq \Bigl| \mathrm{Fib}(n) - \frac{ \varphi^n } {\sqrt{5}} \Bigr|
        + \Bigl| \frac{ \varphi^n } {\sqrt{5}} - m \Bigr| \\
        & < \frac{1}{2} + \frac{1}{2} = 1,
\end{align*}
which is a contradiction. Since \(\mathrm{Fib}(n) \neq m\) and both are integers their difference
must be strictly greater than \(1\).

\hfill\(\square\)
\end{document}